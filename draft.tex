\documentclass[12pt,openany]{book}

% Clickable links
% \usepackage[hidelinks]{hyperref}
\usepackage{setspace}
\usepackage{graphicx}

\usepackage[
  colorlinks=true,
  linkcolor=black,
  citecolor=black,
  urlcolor=blue
]{hyperref}


\usepackage{paracol}
\usepackage[T1]{fontenc}
\usepackage[utf8]{inputenc}
\usepackage{textcomp}
\usepackage[english,french]{babel}
\usepackage{setspace}
\usepackage{csquotes}
% Disable numbering of chapters/sections (hide "Chapitre 1, 2, ...")
\setcounter{secnumdepth}{-1}


% Command for back-to-TOC link
\newcommand{\backtotoc}{%
  \par\bigskip
  \noindent\hyperref[toc]{\emph{← Back to Table of Contents}}
}

\begin{document}
% User note: "I've said that before"

\frontmatter
\phantomsection
\label{toc}
\tableofcontents

\mainmatter



\chapter{Gender, Race and Power in Music: From Stage to Society}

\section{Abstract}
This course investigates how musical works construct, negotiate, and challenge cultural identities and gendered power dynamics in music, across historical periods, with a particular focus on the \href{https://en.wikipedia.org/wiki/Aix-en-Provence_Festival}{Aix-en-Provence Festival}, a leading stage for innovative production, and the work of its new general director, Ted Huffman. 


Through close analysis of musical rhetoric (?) and staging practices, students examine how opera and musical theatre have shaped cultural narratives about femininity, masculinity, and authority. The course situates opera within broader social debates on identity, representation, and agency. Students also explore intersections of class, race, and sexuality as they appear in operatic storytelling. 


\chapter{Ideas}

\emph{Every performance is a political act.}



\section{graphs}

\subsection{bourdieu charts}


\begin{figure}[h]
  \centering
  \includegraphics[width=0.35\textwidth]{images/chanson.jpg}
  \caption{Chanson (inserted image)}
  \label{fig:chanson}
\end{figure}



\begin{figure}[h]
  \centering
  \includegraphics[width=0.35\textwidth]{images/chanson2.jpg}
  \caption{visualization}
\end{figure}
% notes:


\begin{figure}[h]
  \centering
  \includegraphics[width=0.35\textwidth]{images/bourdieu-visu.jpg}
  \caption{bourdieu's data visualization of social space in datawrapper}
  \label{bourdieu}
\end{figure}


\begin{figure}[h]
  \centering
  \includegraphics[width=0.35\textwidth]{images/tableau4.jpg}
  \caption{compositeurs connus}
\end{figure}


\begin{figure}[h]
  \centering
  \includegraphics[width=0.35\textwidth]{images/compositeurs.jpg}
  \caption{compositeurs connus}
\end{figure}



\begin{figure}[h]
  \centering
  \includegraphics[width=0.35\textwidth]{images/musicindustry.jpg}
  \caption{Music industry in \href{https://www.ifpi.org/wp-content/uploads/2024/03/GMR2025_SOTI.pdf}{Global Music Report}}
  \label{fig:musicindustry}
\end{figure}











\backtotoc




\section{Sociology of Music sync}
\href{https://en.wikipedia.org/wiki/Sociology_of_music}{Sociology of music}

\subsection{\href{https://www.oxfordbibliographies.com/display/document/obo-9780199756384/obo-9780199756384-0198.xml}{oxford-bibliographies-soc-mus}}
bla


\subsection{what is sociological about music? William G. Roy}
How does music relate to broader social distinctions, especially class, race, and gender

Try voice tapping to see how it works


p.26 : How Does Music Relate to Broader Social Distinctions?


P.33 : however, another segment is marked by its passing familiarity with a wide range of musical genres liked by less privileged groups (van Eijck 2001). The latter segment's \textit{omnivorous} tastes complicate stratification because, for a socially advantaged group of listeners, the alignment between social
33
and symbolic boundaries is more ``heterologous'' than Bourdieu's argument suggests (García-Álvarez et al. 2007). Even in France, recent survey results ``radically eliminate any attempt to map the distribution of musical taste in terms of \ldots{} homology: highbrow is no more music of the upper-class than pop music the music of the lower class'' (Coulangeon \& Lemel 2007: 98-99). However, these omnivorous tastes in musical genres do not mark the end of stratification by any means. Instead, they appear to represent a new form of currency that the advantaged can deploy in highly individualized ways (Ollivier 2008, Savage 2006, Warde \& Gayo-Cal 2009).


\section{Economics of Aix-en-Provence Festival}
\href{https://www.lefigaro.fr/conjoncture/2017/07/02/20002-20170702ARTFIG00104-au-festival-d-aix-les-spectateurs-ne-paient-que-165-du-cout-de-leur-billet.php}{economics of the festival}   

In 2019, the festival had a budget of around €22 million, of which €8 million subsidies[6] and more than 60/100 self-financed by ticketing and sponsoring

\section{Sociology of culture}
\begin{itemize}
  \item \href{https://en.wikipedia.org/wiki/Cultural_studies}{Cultural studies wikipedia}
  \item \href{https://www.mit.edu/~allanmc/bourdieu1.pdf}{\textit{Distinction: A Social Critique of the Judgement of Taste}, Pierre Bourdieu}
  \item \href{https://mega.nz/file/8t4RiK4b#0J1cM_aSMbCkVU4Ek6MUc3TuuuoaWBjR3mkwI1hrPYg}{Distincion Original French version}
  \item (check bilingual in this folder)
  \item \href{https://archive.org/details/introducingcultu0000sard/page/6/mode/2up}{\textit{Introducing Cultural Studies}, Ziauddin Sardar and Borin Van Loon}
\end{itemize}

\section{Sociology of gender}
\begin{itemize}
  \item \href{https://www.youtube.com/watch?v=UD9IOllUR4k}{Judith Butler explains gender theory}

\end{itemize}






\section{From Stage to Society: Opera as Social Critique: Brecht/Weill and Offenbach}

\textit{Die Dreigroschenoper} (\textit{The Threepenny Opera}) by Bertolt Brecht and Kurt Weill, along with Jacques Offenbach’s \textit{Orphée aux enfers} (\textit{Orpheus in the Underworld}), exemplify the ways in which the musical stage functions as a space of social critique, where relations of class, gender, and power are staged, exaggerated, and made visible. Far from being limited to an aesthetic or purely entertaining function, these works transform opera into a critical apparatus that connects theatrical performance to the social structures organizing collective life.

In \textit{Die Dreigroschenoper}, \textbf{the Brechtian strategy of alienation (\textit{Verfremdungseffekt}) prevents any naïve identification between the audience and the characters, }instead encouraging spectators to adopt an analytical stance toward the social mechanisms on display. Crime, marriage, respectability, and economic survival appear as practices structured by capitalism, revealing the homology between marginal and bourgeois worlds. This logic resonates with Pierre Bourdieu’s theory, according to which cultural fields reproduce and legitimize relations of domination through symbolic forms. Weill’s music, drawing on cabaret, jazz, and popular song, deliberately blurs the boundary between ``legitimate” and popular culture, thereby challenging the hierarchy of taste and the presumed aesthetic neutrality of opera.
This logic resonates with Pierre Bourdieu's theory, according to which cultural fields reproduce and legitimize relations of domination through symbolic forms. Weill's music, drawing on cabaret, jazz, and popular song, deliberately blurs the boundary between ``legitimate'' and popular culture, thereby challenging the hierarchy of taste and the presumed aesthetic neutrality of opera.

Female characters—most notably Polly Peachum and Jenny—occupy a central position in this critique. Their social roles illuminate the intersections of gender, class, and economy, a core concern of feminist musicology. Far from serving as mere secondary figures, they embody the structural constraints imposed on women within a system in which moral value is subordinated to market value. The stage thus becomes a site where gender norms are negotiated and exposed, revealing their constructed and historically contingent nature.

	\textit{Orphée aux enfers} by Offenbach, for its part, deploys satire to subvert the dominant discourses of the Second Empire. By transforming the gods into frivolous, self-interested, and corrupt figures, the work ridicules political and moral authority, exposing what Bourdieu would describe as symbolic violence exercised by elites through myth and legitimizing narratives. \textbf{The famous ``galop infernal,'' often associated with transgressive collective energy, introduces onto the operatic stage forms of bodily pleasure and social disorder traditionally excluded from serious opera, thereby opening a space for contesting norms of respectability.}

In both works, the movement from stage to society operates through a system of critical mirrors. Musical theater functions as a social laboratory in which relations of domination are stylized and amplified in order to reveal their underlying logics more clearly. By mobilizing Brechtian alienation, Bourdieusian critiques of cultural hierarchy, and insights from feminist musicology, \textit{Die Dreigroschenoper} and \textit{Orphée aux enfers} demonstrate that opera is a profoundly political space—one that not only reflects society but also actively participates in questioning its symbolic and social foundations.

\subsection{Rise and Fall of the City of Mahagonny - Brecht Kurt Weill}
Banned by the Nazi Party

\href{https://en.wikipedia.org/wiki/Rise_and_Fall_of_the_City_of_Mahagonny}{Rise and Fall of the City of Mahagonny}
Add a ref to the \href{https://www.youtube.com/watch?v=T_d_VJbYAfc}{whiskey bar - doors alabama song?}

Mahagony - Kurt Weill and the doors


Among Brecht’s friends were members of the Dadaist group, who aimed at destroying what they condemned as the false standards of bourgeois art through derision and iconoclastic satire. The man who taught him the elements of Marxism in the late 1920s was Karl Korsch



\backtotoc


\section{Miscellaneous}
\begin{itemize}
    \item The students use \LaTeX{} exclusively to write an academic article.
    \item During the class, a blind test is organized to assess who can recognize different pieces of music.
    \item John demonstrates superior knowledge by correctly identifying the artist, the song, the album, and the year of release.

\item chinese opera -

\item les noces - compte volage - 

\item travesti - ted huffman - 

\item opera transgenre -

\item \href{https://www.bilibili.com/video/BV1X4QbYBEWH/?spm_id_from=333.337.search-card.all.click&vd_source=06ea9d1e2fb21968b3028363bc959ba5}{Old Snow White} \href{https://en.wikipedia.org/wiki/Snow_White_(2025_film)}{New Snow White }

\item students in literature, media studies, history, or cultural studies.

 \item Medium is the message - McLuhan?


Castrated singers in opera - Farinelli



  \end{itemize}



\backtotoc




\section{other abstracts?}

\subsection{Gender, Race, and Power in Music: From Stage to Society}

This course investigates how operatic works construct, negotiate, and challenge gendered power dynamics across historical periods. Through close analysis of libretti, musical rhetoric, and staging practices, students examine how opera has shaped cultural narratives about femininity, masculinity, and authority.

Special attention is given to the political implications of vocal types, character archetypes, and performance conventions. The course situates opera within broader social debates on identity, representation, and agency. Students explore intersections of class, race, and sexuality as they manifest in operatic storytelling.

Contemporary productions are analyzed for their reimagining of inherited power structures. Multimedia resources highlight evolving directorial strategies that foreground gender critique. By the end of the course, students gain tools for interpreting opera as both an artistic form and a sociocultural force.

\bigskip
\emph{Every performance is a political act.}

\backtotoc

\subsection{Gender and Power in Music: From Stage to Society}

This course investigates how operatic works construct, negotiate, and challenge gendered power dynamics across historical periods. Through close analysis of libretti, musical rhetoric, and staging practices, students examine how opera has shaped cultural narratives about femininity, masculinity, and authority.

The course situates opera within broader social debates on identity, representation, and agency. Students explore intersections of class, race, and sexuality as they manifest in operatic storytelling.

\backtotoc

\subsection{Cultural Voices: Gender and Storytelling in Music and Media}

This course explores how gender shapes narrative strategies across music, opera, and contemporary media. Students investigate how storytellers use sound, voice, and visual framing to construct cultural meanings around identity and power.

Operatic case studies draw particular attention to the Aix-en-Provence Opera Festival as a site of innovative production.

\backtotoc

\subsection{Music, Identity, and the Politics of Performance}

This course explores how the operatic stage serves as a powerful arena for the construction and contestation of identity. Music, libretto, and staging are examined as forces shaping representations of nationality, race, gender, and class.

Ultimately, the course presents opera as a living art form deeply engaged in pressing cultural dialogues.

\backtotoc

\subsection{Staging Power: A History of Political Expression in Music and Theatre}
 How have societies used musical drama to wield power? This course answers that question through a historical survey from ancient Greek civic theatre to contemporary popular performance.

 We examine moments where politics and performance converge, including medieval religious drama, absolutist court spectacle, nationalist opera, and modern protest music. Emphasis is placed on the visual, spatial, and institutional contexts of performance, revealing a shared history of artistic and political expression.

\backtotoc

\subsection{Gender and Power in Music: From Stage to Society}

This course examines how operatic and musical traditions construct, challenge, and negotiate gendered identities. Students analyze representations of femininity, masculinity, and authority across historical and contemporary works.

Attention is given to vocal typologies, character archetypes, and staging practices as sites of \textbf{ideological meaning. Case studies include canonical works and modern adaptations, with particular focus on the Aix-en-Provence Festival and the work of director Ted Huffman.}

Students engage with feminist, queer, and intersectional critiques, developing critical scholarly insight into how music mirrors and shapes the politics of gender in society.

\backtotoc

\subsection{Gender and Power in Music: From Stage to Society II}

 This course explores the dynamic interplay between gender, power, and music with a focus on operatic representation. Through analysis of vocal range, orchestration, and libretto, students examine archetypes such as the tragic heroine, the fallen hero, and the authoritarian figure.

 Moving beyond the score, the course considers power relations behind the scenes among composers, patrons, and performers. From Baroque spectacle to contemporary reinterpretation, students explore how feminist and queer perspectives reshape canonical works.

\backtotoc

\subsection{Opera and Beyond: Gender, Performance, and Power}

This course explores how opera stages questions of gender, identity, and power from its origins to modern performance.

\backtotoc



\section{data}

\href{https://www.datawrapper.de/}{https://www.datawrapper.de/}
\begin{itemize}


    \item \href{https://dataverse.harvard.edu/}{Harvard Dataverse}
  \item \href{https://www.worldvaluessurvey.org}{World Values Survey (WVS)}
  
  \item \href{https://www.europeansocialsurvey.org}{European Social Survey (ESS)}
  
  \item \href{https://gss.norc.org}{General Social Survey (GSS -- USA)}
  
  \item \href{https://www.icpsr.umich.edu}{ICPSR (Inter-university Consortium for Political and Social Research)}
  
  \item \href{https://ukdataservice.ac.uk}{UK Data Service}
  
  \item \href{https://ec.europa.eu/eurostat}{Eurostat -- Cultural Statistics}
\end{itemize}


\backtotoc



\chapter{Sociology}



\section{Some sociologists (of culture)}

\subsection{Swidler}

\href{https://www.youtube.com/@sociologyclass7600/videos}{Ann Swidler's videos}
\href{https://youtu.be/52GUxpqKUlE?list=PLktOMviQNUf5d899hRAFS5OWCEfCJGKmR}{Swidler}


\subsection{Geertz}
Clifford Geertz on
balinese cockfight
\href{https://en.wikipedia.org/wiki/Deep_Play:_Notes_on_the_Balinese_Cockfight}{Cockfight}

\subsection{Harvey Molotch - NYU}
\href{https://youtu.be/4FduU3EokBY}{NYU}

\subsubsection{Deviance}
\href{https://youtu.be/s_-NlN8BjwM}{deviance}


\subsection{Sociology Father}
\href{https://youtu.be/hd33BahdAjs?list=PL37FC9556148B7E1F}{Szelenyi}


\subsection{Harvard Pawan Dhingra}
\href{https://youtu.be/H3W-PDIlUsE}{Pawan Dhingra}

\section{Bourdieu}

\subsection{Cultural misery?}

\href{https://fr.wikipedia.org/wiki/La_Distinction#Critiques_de_l'ouvrage}{misabilisme culturel}

A working-class listener who enjoys popular music may:
\begin{itemize}
  \item feel their taste is ``bad'' or ``not real music'';
  \item remain silent in academic or elite settings;
  \item defer to critics or institutions they do not fully understand.
\end{itemize}

They are not incapable of aesthetic judgment —
they are denied the authority to judge.

That denial is cultural misery.



\subsection{Bourdieu bilingual}


% \setlength{\parindent}{1.5em}
% \onehalfspacing


% \chapter*{Introduction}

\begin{paracol}{2}

% ---------- LEFT COLUMN: ENGLISH ----------
\selectlanguage{english}


\begin{quote}
"You said it, my good knight! There ought to be laws to protect the body of acquired knowledge. Take one of our good pupils, for example: modest and diligent, from his earliest grammar classes he's kept a little notebook full of phrases. After hanging on the lips of his teachers for twenty years, he's managed to build up an intellectual stock in trade; doesn't it belong to him as if it were a house, or money? Paul Claudel, Le soulier de satin, Day III, Scene ii "
\end{quote}


 There is an economy of cultural goods, but it has a specific logic. 
 \vspace{3em}




 Sociology endeavours to establish the conditions in which the consumers of cultural goods, and their taste for them, are produced, and at the same time to describe the different ways of appropriating such of these objects as are regarded at a particular moment as works of art, and the social conditions of the constitution of the mode of appropriation that is con sidered legitimate. But one cannot fully understand cultural practices unless 'culture', in the restricted, normative sense of ordinary usage, is brought back into ' culture' in the anthropological sense, and the elabo rated taste for the most refined objects is reconnected with the elemen tary taste fo r the flavours of food. 
 
 
 
 Whereas the ideology of charisma regards taste in legitimate culture as a gift of nature, scientific observation shows that cultural needs are the product of upbringing and education: surveys establish that all cultural practices (museum visits, concert-going, reading etc.), and preferences in literature, painting or music, are closely linked to educational level (measured by qualifications or length of schooling) and secondarily to social origin. 
 
 \vspace{3em}
The relative weight of home background and of formal education (the effectiveness and duration of which are closely dependent on social origin) varies according to the extent to which the different cultural practices are recognized and taught by the educational system, and the influence of social origin is strongest-other things being equal-in 'extra-curricular' and avant-garde culture. 



To the socially recog nized hierarchy of the arts, and within each of them, of genres, schools or periods, corresponds a social hierarchy of the consumers. This predisposes 2 / Introduction tastes to function as markers of 'class'. The manner in which culture has been acquired lives on in the manner of using it: the importance attached to manners can be understood once it is seen that it is these imponder ables of practice which distinguish the different-and ranked-modes of culture acquisition, early or late, domestic or scholastic, and the classes of individuals which they characterize (such as 'pedants' and mondains). Culture also has its titles of nobility-awarded by the educational system-and its pedigrees, measured by seniority in admission to the nobility. The definition of cultural nobility is the stake in a struggle which has gone on unceasingly, from the seventeenth century to the present day, between groups differing in their ideas of culture and of the legitimate relation to culture and to works of art, and therefore differing in the conditions of acquisition of which these dispositions are the product 2 Even in the classroom, the dominant definition of the legitimate way of appropriating culture and works of art favours those who have had early access to legitimate culture, in a cultured household, outside of scholastic disciplines, since even within the educational system it devalues scholarly knowledge and interpretation as 'scholastic' or even 'pedantic' in favour of direct experience and simple delight. The logic of what is sometimes called, in typically 'pedantic' language, the 'reading' of a work of art, offers an objective basis for this opposition. Consumption is, in this case, a stage in a process of communication, that is, an act of deciphering, decoding, which presupposes practical or ex plicit mastery of a cipher or code. In a sense, one can say that the capacity to see (voir) is a function of the knowledge (savoir), or concepts, that is, the words, that are available to name visible things, and which are, as it were, programmes for perception. A work of art has meaning and interest only for someone who possesses the cultural competence, that is, the code, into which it is encoded. The conscious or unconscious implemen tation of explicit or implicit schemes of perception and appreciation which constitutes pictorial or musical culture is the hidden condition fo r recognizing the styles characteristic of a period, a school or an author, and, more generally, for the familiarity with the internal logic of works that aesthetic enjoyment presupposes. A beholder who lacks the specific code feels lost in a chaos of sounds and rhythms, colours and lines, with out rhyme or reason. Not having learnt to adopt the adequate disposi tion, he stops short at what Erwin Panofsky calls the 'sensible properties', perceiving a skin as downy or lace-work as delicate, or at the emotional resonances aroused by these properties, referring to 'austere' colours or a 'joyful' melody. He cannot move from the 'primary stratum of the meaning we can grasp on the basis of our ordinary experience' to the 'stratum of secondary meanings', i.e., the 'level of the meaning of what is signified', unless he possesses the concepts which go beyond the sensible properties and which identify the specifically stylistic properties of the Introduction / 3 work. 3 Thus the encounter with a work of art is not 'love at first sight' as is generally supposed, and the act of empathy, Einfiihlung, which is the art-lover's pleasure, presupposes an act of cognition, a decoding opera tion, which implies the implementation of a cognitive acquirement, a cultural code. 4 This typically intellectualist theory of artistic perception directly con tradicts the experience of the art-lovers closest to the legitimate defini tion; acquisition of legitimate culture by insensible familiarization within the family circle tends to favour an enchanted experience of culture which implies forgetting the acquisition. 5 The 'eye' is a product of his tory reproduced by education. This is true of the mode of artistic percep tion now accepted as legitimate, that is, the aesthetic disposition, the capacity to consider in and for themselves, as form rather than function, not only the works designated for such apprehension, i.e., legitimate works of art, but everything in the world, including cultural objects which are not yet consecrated-such as, at one time, primitive arts, or, nowadays, popular photography or kitsch-and natural objects. The 'pure' gaze is a historical invention linked to the emergence of an auton omous field of artistic production, that is, a field capable of imposing its own norms on both the production and the consumption of its prod ucts. 6 An art which, like all Post-Impressionist painting, is the product of an artistic intention which asserts the primacy of the mode of representa tion over the object of representation demands categorically an attention to fo rm which previous art only demanded conditionally. The pure intention of the artist is that of a producer who aims to be autonomous, that is, entirely the master of his product, who tends to re ject not only the 'programmes' imposed a priori by scholars and scribes, but also-following the old hierarchy of doing and saying-the interpre tations superimposed a posteriori on his work. The production of an 'open work', intrinsically and deliberately polysemic, can thus be under stood as the final stage in the conquest of artistic autonomy by poets and, following in their footsteps, by painters, who had long been reliant on writers and their work of 'showing' and 'illustrating'. To assert the au tonomy of production is to give primacy to that of which the artist is master, i.e., form, manner, style, rather than the 'subject', the external ref erent, which involves subordination to functions-even if only the most elementary one, that of representing, signifying, saying something. It also means a refusal to recognize any necessity other than that inscribed in the specific tradition of the artistic discipline in question: the shift from an art which imitates nature to an art which imitates art, deriving from its own history the exclusive source of its experiments and even of its breaks with tradition. An art which ever increasingly contains refer ence to its own history demands to be perceived historically; it asks to be referred not to an external referent, the represented or designated 'reality', but to the universe of past and present works of art. Like artistic production, in that it is generated in a field, aesthetic perception is necessarily historical, inasmuch as it is differential, relational, attentive to the devia tions (ecarts) which make styles. Like the so-called naive painter who, operating outside the field and its specific traditions, remains external to the history of the art, the 'naive' spectator cannot attain a specific grasp of works of art which only have meaning or value-in relation to the specific history of an artistic tradition. The aesthetic disposition de manded by the products of a highly autonomous field of production is inseparable from a specific cultural competence. This historical culture functions as a principle of pertinence which enables one to identify. among the elements offered to the gaze, all the distinctive features and only these, by referring them, consciously or unconsciously, to the uni verse of possible alternatives. This mastery is, fo r the most part, acquired simply by contact with works of art-that is, through an implicit learn ing analogous to that which makes it possible to recognize fa miliar faces without explicit rules or criteria-and it generally remains at a practical l evel; it is what makes it possible to identify styles, i.e., modes of expres sion characteristic of a period, a civilization or a school, without having to distinguish clearly, or state explicitly, the features which constitute their originality. Everything seems to suggest that even among profes sional valuers, the criteria which define the stylistic properties of the 'typ ical works' on which all their judgements are based usually remain implicit. The pure gaze implies a break with the ordinary attitude towards the world, which, given the conditions in which it is performed, is also a so cial separation. Ortega y Gasset can be believed when he attributes to modern art a systematic refusal of all that is 'human', i.e .. generic, com mon-as opposed to distinctive. or distinguished-namely. the passions, emotions and feelings which 'ordinary' people invest in their 'ordinary' lives. It is as if the 'popular aesthetic' (the quotation marks are there to indicate that this is an aesthetic 'in itself' not 'for itself') were based on the affirmation of the continuity between art and life, which implies the subordination of form to function. This is seen clearly in the case of the novel and especially the theatre, where the working-class audience refuses any sort of formal experimentation and all the effects which, by intro ducing a distance from the accepted conventions (as regards scenery, plot etc.), tend to distance the spectator, preventing him from getting in volved and fully identifying with the characters (I am thinking of Brechtian 'alienation' or the disruption of plot in the nouveau roman). In contrast to the detachment and disinterestedness which aesthetic theory regards as the only way of recognizing the work of art for what it is, i.e., autonomous, selbstiindig, the 'popular aesthetic' ignores or refuses the re fusal of 'facile' involvement and 'vulgar' enjoyment, a refusal which is the basis of the taste fo r formal experiment. And popular judgements of paintings or photographs spring from an 'aesthetic' (in fact it is an Introduction / 5 ethos) which is the exact opposite of the Kantian aesthetic. Whereas, in order to grasp the specificity of the aesthetic judgement, Kant strove to distinguish that which pleases from that which gratifies and, more gen erally, to distinguish disinterestedness, the sole guarantor of the specifi cally aesthetic quality of contemplation, from the interest of reason which defines the Good, working-class people expect every image to ex plicitly perform a function, if only that of a sign, and their judgements make reference, often explicitly, to the norms of morality or agreeable ness. Whether rejecting or praising, their appreciation always has an eth ical basis. Popular taste applies the schemes of the ethos, which pertain in the or dinary circumstances of life, to legitimate works of art, and so performs a systematic reduction of the things of art to the things of life. The very seriousness (or naivety) which this taste invests in fictions and represen tations demonstrates a contrario that pure taste performs a suspension of 'naive' involvement which is one dimension of a 'quasi-ludic' relation ship with the necessities of the world. Intellectuals could be said to be lieve in the representation-literature, theatre, painting-more than in the things represented, whereas the people chiefly expect representations and the conventions which govern them to allow them to believe 'na ively' in the things represented. The pure aesthetic is rooted in an ethic, or rather, an ethos of elective distance from the necessities of the natural and social world, which may take the form of moral agnosticism (visible when ethical transgression becomes an artistic parti pris) or of an aesthet icism which presents the aesthetic disposition as a universally valid prin ciple and takes the bourgeois denial of the social world to its limit. The detachment of the pure gaze cannot be dissociated from a general dispo sition towards the world which is the paradoxical product of condition ing by negative economic necessities-a life of ease-that tends to induce an active distance from necessity. Although art obviously offers the greatest scope to the aesthetic dispo sition, there is no area of practice in which the aim of purifying, refining and sublimating primary needs and impulses cannot assert itself, no area in which the stylization of life, that is, the primacy of fo rms over func tion, of manner over matter, does not produce the same effects. And nothing is more distinctive, more distinguished, than the capacity to confer aesthetic status on objects that are banal or even 'common' (be cause the 'commen' people make them their own, especially for aesthetic purposes), or the ability to apply the principles of a 'pure' aesthetic to the most everyday choices of everyday life, e.g., in cooking, clothing or deco ration, completely reversing the popular disposition which annexes aes thetics to ethics. In fact, through the economic and social conditions which they pre suppose, the different ways of relating to realities and fictions, of believ ing in fictions and the realities they simulate, with more or less distance 6 / Introduction and detachment, are very closely linked to the different possible positions in social space and, consequently, bound up with the systems of disposi tions (habitus) characteristic of the different classes and class fractions. Taste classifies, and it classifies the classifier. Social subjects, classified by their classifications, distinguish themselves by the distinctions they make, between the beautiful and the ugly, the distinguished and the vulgar, in which their position in the objective classifications is expressed or be trayed. And statistical analysis does indeed show that oppositions similar in structure to those found in cultural practices also appear in eating habits. The antithesis between quantity and quality, substance and form, corresponds to the opposition-linked to different distances from neces sity--between the taste of necessity, which favours the most 'filling' and most economical foods, and the taste of liberty-or luxury-which shifts the emphasis to the manner (of presenting, serving, eating etc.) and tends to use stylized forms ro deny function. The science of taste and of cultural consumption begins with a trans gression that is in no way aesthetic: it has to abolish the sacred frontier which makes legitimate culture a separate universe, in order to discover the intelligible relations which unite apparently incommensurable 'choices', such as preferences in music and food, painting and sport, liter ature and hairstyle. This barbarous reintegration of aesthetic consump tion into the world of ordinary consumption abolishes the opposition, which has been the basis of high aesthetics since Kant, between the 'taste of sense' and the 'taste of reflection', and between facile pleasure, pleasure reduced to a pleasure of the senses, and pure pleasure, pleasure purified of pleasure, which is predisposed to become a symbol of moral excellence and a measure of the capacity for sublimation which defines the truly human man. The culture which results from this magical division is sa cred. Cultural consecration does indeed confer on the objects, persons and situations it touches, a sort of ontological promotion akin to a tran substantiation. Proof enough of this is found in the two following quo tations, which might almost have been written for the delight of the sociologist: 'What struck me most is this: nothing could be obscene on the stage of our premier theatre, and the ballerinas of the Opera, even as naked dancers, sylphs, sprites or Bacchae, retain an inviolable purity.'7 - 'There are obscene postures: the stimulated intercourse which offends the eye. Clearly, it is impossible to approve, although the interpolation of such gestures in dance routines does give them a symbolic and aesthetic quality which is absent from the intimate scenes the cinema daily flaunts before its spectators' eyes . . . As for the nude scene, what can one say, except that it is brief and theatrically not very effective? I will not say it is chaste or innocent, for nothing commercial can be so described. Let us say it is not shocking, and that the chief objection is that it serves as a box-office gimmick. . . . In Hair, the nakedness fails to be symbolic.'8 Introduction / 7 The denial of lower, coarse, vulgar, venal, servile-in a word, oatu ral--enjoyment, which constitutes the sacred sphere of culture, implies an affirmation of the superiority of those who can be satisfied with the sublimated, refined, disinterested, gratuitous, distinguished pleasures fo r ever closed to the profane. That is why art and cultural consumption are predisposed, consciously and deliberately or not, to fulfil a social function of legitimating social differences. 
% --- continue pasting English text here ---

\switchcolumn

% ---------- RIGHT COLUMN: FRENCH ----------
\selectlanguage{french}

\begin{quote}
 
 «Vous l’avez dit, cavalier ! il devrait y avoir des lois  pour protéger les connaissances acquises. Prenez un  de nos  bons élèves par  exemple, modeste diligent, qui dès ses  classes de grammaire a commencé à tenir son petit cahier d’expressions, Qui pendant vingt années suspendu aux lèvres de ses  professeurs a fini par  se  composer une espèce de petit pécule intellectuel : est-ce qu’il ne lui appartient pas  comme si c’était une maison ou de l’argent ? P. Claudel, soulier de satin »
\end{quote}

 Il y a  une économie des  biens culturels, mais cette économie a une logique spécifique qu’il faut dégager pour échapper à l’économisme. 
 
 Cela en  travaillant d’abord à établir les  conditions dans lesquelles sont produits les consommateurs de biens culturels et leur goût, en même temps qu’à décrire les  différentes manières de  s’approprier ceux d’entre ces  biens qui sont considérés à un  moment donné du temps comme des oeuvres d’art et  les  conditions sociales de  la constitution du  mode d’appropriation qui  est  tenu pour légitime. 
 
 \vspace{5em}
 Contre l’idéologie charismatique qui  tient les  goûts en  matière de  culture légitime pour un  don de  la nature, l’observation scientifique montre que les  besoins culturels sont le  produit de l'éducation : l’enquête établit que toutes les  pratiques culturelles (fréquentation des musées, des  concerts, des  expositions, lecture, etc.) et  les  préférences en  matière de littérature, de peinture ou  de musique, sont étroitement liées au  niveau d’instruction (mesuré au titre scolaire ou  au  nombre d’années d’études), et secondairement à l’origine sociale (1). 
 
 
 Le  poids relatif de  l’éducation familiale et de  l’éducation proprement scolaire (dont l'efficacité et  la  durée dépendent étroitement de  l’origine sociale) varie selon le  degré auquel les différentes pratiques culturelles sont reconnues et enseignées par le système scolaire, et l’influence de l’origine sociale n’est jamais aussi forte, toutes choses étant égales par ailleurs, qu’en matière de  «culture libre» ou  de  culture d’avant-garde. 
 
 
 A la hiérarchie socialement reconnue des arts et, à l’intérieur de chacun d’eux, des genres, des écoles ou des époques, correspond la hiérarchie sociale des consommateurs. Ce  qui prédispose les  goûts à fonctionner comme des  marqueurs privilégiés de  la «classe». Les manières d’acquérir se survivent dans la manière d’utiliser les acquis : l'attention accordée aux  manières s’explique si l’on voit que c’est à ces  impondérables de la pratique que se  reconnaissent les différents modes d’acquisition, hiérarchisés, de  la culture, précoce ou  tardif, familial ou  scolaire, et  les  classes d’individus qu’elles caractérisent (comme les  «pédants» et  les  «mondains»). La  noblesse culturelle a  aussi ses  titres, que décerne l’école, et  ses  quartiers, que mesure l’ancienneté de l’accès à la noblesse. La  définition de  la noblesse culturelle est l’enjeu d’une lutte qui, du  XVIIe siècle à nos jours, n’a  cessé d’opposer, de manière plus ou  moins déclarée, des  groupes séparés dans leur idée de  la culture, du  rapport légitime à  la  culture et  aux oeuvres d’art, donc dans les conditions d’acquisition dont ces  dispositions sont le produit : la définition dominante du  mode d’appropriation légiti- me  de  la culture et  de  l’oeuvre d’art favorise, jusque sur  le terrain scolaire, ceux qui ont eu  accès à la culture légitime très tôt, dans une famille cultivée, hors des  disciplines scolaires ; elle dévalue en effet le savoir et l’interprétation savante , marquée comme «scolai- re», voire «pédante», au  profit de  l’expérience directe et  de  la simple délectation. La  logique de  ce  que l’on appelle parfois, dans un  langage typiquement «pédant», la  «lecture» de  l’oeuvre d’art, offre un fondement objectif à cette opposition. L’oeuvre d’art ne  prend un sens et  ne  revêt un  intérêt que pour celui qui est  pourvu du  code selon lequel elle est codée. La  mise en  oeuvre consciente ou inconsciente du  système de  schèmes de  perception et  d’apprécia- tion plus ou  moins explicites qui  constitue la culture picturale ou musicale est  la  condition cachée de  cette forme élémentaire de connaissance qu'est la  reconnaissance des styles. Le  spectateur dépourvu du  code spécifique se  sent submergé, «noyé», devant ce qui lui apparaît comme un  chaos de sons et de rythmes, de couleurs et  de  lignes sans rime ni raison. Faute d’avoir appris à adopter la disposition adéquate, il s’en tient à ce  que Panofsky appelle les «propriétés sensibles», saisissant une peau comme veloutée ou une dentelle comme vaporeuse, ou  aux résonances affectives suscitées par ces  propriétés, parlant de  couleurs ou  de  mélodies sévères ou  joyeuses. On ne  peut en  effet passer de  la  «couche primaire du  sens que nous pouvons pénétrer sur  la base de  notre expérience existentielle» à  la  «couche des sens secondaires», c’est-à-dire à la «région du sens du signifié», que si l’on possède les concepts qui, dépassant les  propriétés sensibles, saisissent les caractéristiques proprement stylistiques de  l’oeuvre (2). C’est 2—E. Panofsky, «Iconography and Iconology : An  Introduction to the  Study je MR PE Art», Meaning in the Visual Arts, New York, Doubleday and 0, 1955, p. 28. 
Introduction III dire que la  rencontre avec l’oeuvre d’art n’a rien du  coup de foudre que l’on veut y voir d’ordinaire et que l’acte de fusion affective, d'Einfühlung, qui fait le plaisir d’amour de l’art, suppose un acte de connaissance, une opération de déchiffrement, de décodage, qui implique la mise en oeuvre d’un patrimoine cognitif, d’une compétence culturelle. Cette théorie typiquement intellectua- liste de la perception artistique contredit très directement l’expé- rience des amateurs les plus conformes à la définition légitime : l’acquisition de la culture légitime par la familiarisation insensible au sein de la famille tend en effet à favoriser une expérience enchantée de la culture qui implique l’oubli de l’acquisition et l’ignorance des instruments de l’appropriation. L'expérience du plaisir esthétique peut aller de pair avec le malentendu ethnocen- trique qu’entraîne l’application d’uncodeimpropre. Ainsi, le regard «pur» que porte sur les oeuvres le spectateur cultivé d’aujourd’hui n’a à peu près rien de commun avec l’«oeil moral et spirituel» des hommes du Quattrocento, c’est-à-dire l’ensemble des disposi- tions à la fois cognitives et évaluatives qui étaient au principe de leur perception du monde et de leur perception de la représentation picturale du monde : soucieux, comme le montrent les contrats, d’en avoir pour leur argent, les clients des Filippo Lippi, Domenico Ghirlandaio ou Piero della Francesca investissaient dans les oeuvres d’art les dispositions mercantiles d'hommes d’affaires rompus au calcul immédiat des quantités et des prix, recourant par exemple à des critères d’appréciation tout à fait surprenants, comme la cherté des couleurs —qui place l’or et le bleu d’outremer au sommet de la hiérarchie— (3). L’«oeil» est un produit de l’histoire reproduit par l’éducation. Il  en  est  ainsi du  mode de  perception artistique qui s'impose aujourd’hui comme légitime, c’est-à-dire la disposition esthétique comme capacité de considérer en elles-mêmes et pour elles-mêmes, dans leur forme et non dans leur fonction, non seulement les oeuvres désignées pour une telle appréhension, c’est-à-dire les oeuvres d’art légitimes, mais toutes les choses du monde, qu'il s’agisse des oeuvres culturelles qui ne sont pas encore consacrées —-comme, en un temps, les arts primitifs ou, aujourd’hui, la photographie populaire ou le kitsch— ou des objets naturels. Le regard «pur» est une invention historique qui est corrélative de l’apparition d’un champ de production artistique autonome, c’est-à-dire capable d'imposer ses propres normes tant dans la production que dans la consommation de ses produits (4). Un art qui, comme toute la peinture post- 3-Cf, M.  Baxandall, Painting and Expérience in  Fifteenth Century Italy, À Primer in  the Social History of Pictorial Style, Oxford, Oxford University Press, 1972. 4-—Cf. P.  Bourdieu, Le  marché des  biens symboliques, L'Année sociologique, Vol. 22,  1971, pp.  49-126 ; Éléments d’une théorie sociologique de la percep- tion artistique, Revue internationale des sciences sociales, XX, 4, 1968, pp. 640-664. 
IV  La distinction impressionniste par  exemple, est  le produit d’une intention artisti- que affirmant le primat du mode de représentation sur l’objet de la représentation, exige catégoriquement une attention exclusive à la forme que l’art antérieur n’exigeait que conditionnellement. L’intention pure de  l’artiste est  celle d’un producteur qui se  veut autonome, c’est-à-dire entièrement maître de son produit, qui tend à  récuser non seulement les  «programmes» imposés a priori par les clercs et les lettrés mais aussi, avec la vieille hiérarchie du  faire et  du  dire, les  interprétations surimposées a posteriori sur son  oeuvre : la production d’une «oeuvre ouverte», intrinsèquement et  délibérément polysémique, peut être ainsi comprise comme le dernier stade de  la  conquête de  l’autonomie artistique par les poêtes et,  sans doute à  leur image, par les  peintres, longtemps tributaires des écrivains et  de  leur travail de  «faire-voir» et  de «faire-valoir». Affirmer l’autonomie de  la production, c’est confé- rer  la primauté à ce  dont l'artiste est maître, c’est-à-dire la forme, la manière, le style, par  rapport au  «sujet», référent extérieur, par où  s’'introduit la  subordination à  des fonctions-s’agirait-il de  la plus élémentaire, celle de  représenter, de signifier, de dire quelque chose. C’est du  même coup refuser de  reconnaître aucune autre nécessité que celle qui  se  trouve inscrite dans la tradition propre de  la  discipline artistique considérée ; c’est passer d’un art  qui imite la nature, à un  art  qui imite l’art, trouvant dans son  histoire propre le  principe exclusif de  ses  recherches et  de  ses  ruptures mêmes avec la tradition. Un art  qui enferme toujours davantage la  référence à sa propre histoire appelle un  regard historique ; il  demande à  être référé non à ce  référent extérieur qu’est la «réalité» représentée ou désignée mais à l’univers des  oeuvres d’art du  passé et  du  présent. Comme la production artistique en  tant qu’elle s’engendre dans un champ, la perception esthétique, en  tant qu’elle est  différentielle, relationnelle, attentive aux  écarts qui font les styles, est nécessaire- ment historique : comme le peintre dit  «naïf» qui, étant extérieur au  champ et à ses  traditions spécifiques, reste extérieur à l’histoire propre de  l’art considéré, le  spectateur «naïf» ne  peut accéder à  une perception spécifique d’oeuvres d’art qui n’ont de  sens que par  référence à l’histoire spécifique d’une tradition artistique. La  disposition esthétique qu’appellent les productions d’un champ de  production parvenu à un  haut degré d’autonomie est indissocia- ble  d’une compétence culturelle spécifique : cette culture historique fonctionne comme un  principe de pertinence qui permet de repérer, parmi les  éléments proposés au  regard, tous les  traits distinctifs et ceux-là seulement, en  les référant, plus ou  moins consciemment, à l’univers des  possibilités substituables. Acquise pour l'essentiel par  la simple fréquentation des  oeuvres, c’est-à-dire par un  appren- tissage implicite analogue à  celui qui permet de  reconnaître, sans règles ni critères explicites, des visages familiers, cette maîtrise qui  reste le  plus souvent à l’état pratique, permet de  repérer des 
Introduction V styles, c’est-à-dire des  modes d’expression caractéristiques d’une époque, d’une civilisation ou d’une école, sans que soient claire- ment distingués et  explicitement énoncés les traits qui font l’origi- nalité de  chacun d’eux. Tout semble indiquer que, même chez les professionnels de  l'attribution, les  critères qui définissent les propriétés stylistiques des oeuvres-témoins sur lesquelles s’appuient tous les jugements, restent le plus souvent à l’état implicite. Le  regard pur implique une rupture avec l’attitude ordinaire à l’égard du  monde qui, étant donné les conditions de son  accom- plissement, est une rupture sociale. On peut croire Ortega y Gasset, lorsqu'il attribue à l’art moderne un  refus systématique de  tout ce qui est «humain», c’est-à-dire générique, commun —par opposi- tion à distinctif, ou distingué, à savoir les passions, les émotions, les sentiments que les hommes «ordinaires» engagent dans leur existence «ordinaire». Tout se passe en effet comme si l’«esthéti- que populaire» (les guillemets étant là pour signifier qu’il s’agit d’une esthétique en soi et non pour soi)était fondée sur l'affirmation de la continuité de l’art et de la vie, qui implique la subordination de la forme à la fonction. Cela se voit bien dans le cas du roman et surtout du théâtre où le public populaire refuse toute espèce de recherche formelle et tous les effets qui, en introduisant une distance par rapport aux conventions admises (en matière de décor, d’intrigue, etc.), tendent à mettre le spectateur à distance, l’em- pêchant d’entrer dans le jeu et de s'identifier complètement aux personnages (je pense à la distanciation brechtienne ou à la désarticulation de l'intrigue romanesque opérée par le Nouveau Roman). A l’opposé du détachement, du désintéressement, que la théorie esthétique tient pour la seule manière de reconnaître l’oeuvre d’art pour ce qu’elle est, c’est-à-dire autonome, selbständig, l’«esthétique» populaire ignore ou refuse le refus de l’adhésion «facile» et des abandons «vulgaires» qui est, au moins indirecte- ment, au principe du goût pour les recherches formelles et, comme le disent les jugements populaires sur la peinture ou la photographie, elle se présente comme l’exact opposé de l’esthétique kantienne : pour appréhender ce qui fait la spécificité du jugement esthétique, Kant s’ingéniait à distinguer ce qui plaît de ce qui fait plaisir et, plus généralement, à discerner le désintéressement, seul garant de la qualité proprement esthétique de la contemplation, de l’intérêt de la raison qui définit le Bon ; à l’inverse, les sujets des classes populaires qui attendent de toute image qu’elle remplisse explicite- ment une fonction, fût-ce celle de signe, manifestent dans leurs jugements la référence, souvent explicite, aux normes de la morale ou de l’agrément. Qu'ils blâment ou qu’ils louent, leur appréciation se réfère à un système de normes dont le principe est toujours éthique. En  appliquant aux oeuvres légitimes les  schèmes de l’ethos, qui valent pour les circonstances ordinaires de la vie, et en  opérant ainsi une réduction systématique des  choses de  l’art ou  choses de 
VI  La distinction. la  vie, le  goût populaire et  le sérieux (ou la naïveté) même qu'il investit dans les fictions et les représentations indiquent a confrarlo que le goût pur opère une mise en suspens de l’adhésion «naïve» qui est une dimension d’un rapport quasi ludique avec les nécessités du monde. On pourrait dire que les intellectuels croient à la repré- sentation —littérature, théâtre, peinture— plus qu’aux choses représentées, tandis que le «peuple» demande avant tout aux représentations et aux conventions qui les régissent de lui permettre de croire «naïvement» aux choses représentées. L’esthétique pure s’enracine dans une éthique ou, mieux, un ethos de la distance élective aux nécessités du monde naturel et social qui peut prendre la forme d’un agnosticisme moral (visible lorsque la transgression éthique devient un parti artistique) ou d’un esthétisme qui, en constituant la disposition esthétique en principe d’application universelle, pousse jusqu’à sa limite la dénégation bourgeoise du monde social. On comprend que le détachement du regard pur ne peut être dissocié d’une disposition générale à l’égard du monde qui est le produit paradoxal du conditionnement exercé par des nécessités économiques négatives —ce que l’on appelle les facilités — et propre de ce fait à favoriser la distance active à la nécessité. S’il  est  trop évident que l’art offre à la disposition esthéti- que son terrain par  excellence, il reste qu’il n’est pas  de  domaine de  la  pratique où  ne  puisse s’affirmer l’intention de  soumettre au  raffinement et  à  la  sublimation les  besoins et  les  pulsions primaires, pas de  domaine où  la  stylisation de  la vie, c’est-à-dire le  primat conféré à la  forme sur  la fonction, à la manière sur  la matière, ne  produise les  mêmes effets. Et  rien n’est plus classant, plus distinctif, plus distingué, que la capacité de  constituer esthé- tiquement des objets quelconques ou  même «vulgaires» (parce qu’appropriés, surtout à  des fins esthétiques, par le  «vulgaire») ou  l'aptitude à  engager les  principes d’une esthétique «pure» dans les  choix les  plus ordinaires de  l’existence ordinaire, en matière de  cuisine, de  vêtement ou  de  décoration par exemple, par  une inversion complète de  la disposition populaire qui annexe l'esthétique à l’éthique. En  fait, par l’intermédiaire des conditions économiques et sociales qu’elles supposent, les différentes manières, plus ou  moins détachées ou  distantes, d’entrer en  relation avec les  réalités et les fictions, de  croire aux fictions ou  au  réalités qu’elles simulent, sont très étroitement liées aux différentes positions possibles dans l’espace social et  par là,  étroitement insérées dans les  systèmes de  dispositions (habitus) caractéristiques des différentes classes et  fractions de  classe. Le  goût classe, et  classe celui qui classe : les  sujets sociaux se  distinguent par  les distinctions qu’ils opèrent, entre le  beau et  le laid, le  distingué et  le vulgaire, et où s’exprime ou  se  traduit leur position dans les  classements objectifs. Et de ce fait, l'analyse statistique montre par  exemple que des  oppositions de  même structure que celles qui s’observent en  matière de  con- 
Introduction VII sommations culturelles se  retrouvent aussi en  matière de  consom- mations alimentaires : l’antithèse entre la quantité et la qualité, la grande bouffe et les petits plats, la substance et la forme ou les formes, recouvre l’opposition, liée à des distances inégales à la nécessité, entre le goût de nécessité, qui porte vers les nourritures à la fois les plus nourrissantes et les plus économiques, et le goût de liberté —ou de luxe— qui, par opposition au franc-manger popu- laire, porte à déplacer l’accent de la matière vers la manière (de présenter, de servir, de manger, etc.) par un parti de stylisation qui demande à la forme et aux formes d’opérer une dénégation de la fonction. La  science du  goût et  de  la consommation culturelle com- mence par une transgression qui n’a  rien d’esthétique : elle doit en  effet abolir la  frontière sacrée qui fait de  la  culture légitime un  univers séparé pour découvrir les  relations intelligibles qui unissent des «choix» en  apparence incommensurables, comme les  préférences en  matière de  musique et  de  cuisine, en  matière de  peinture et  de  sport, en  matière de  littérature et  de  coiffure. Cette réintégration barbare des consommations esthétiques dans l’univers des consommations ordinaires révoque l’opposition, qui  est  au  fondement de  l’esthétique savante depuis Kant, entre le «goût des sens» et  le  «goût de  la  réflexion» et  entre le  plaisir «facile», plaisir sensible réduit à un  plaisir des sens, et  le  plaisir «pur», qui est prédisposé à  devenir un  symbole d’excellence morale et  une mesure de  la  capacité de  sublimation qui définit l’homme vraiment humain. La  culture qui  est  le produit de  cette division magique a  valeur de  sacré. Et  de  fait, la  consécration culturelle fait subir aux objets, aux personnes et  aux situations qu’elle touche une sorte de  promotion ontologique qui s’ap- parente à  une transsubstantiation. Je  n’en veux pour preuve que ces deux jugements, qui semblent inventés pour le  bonheur du sociologue :  «Ce qui nous aura frappé le  plus en  définitive : rien ne  saurait être obscène sur  notre première scène, et  les  bal- lerines de  l’Opéra, même en  danseuses nues, sylphes, follets ou bacchantes, gardent une pureté inaltérable» (5)  ; «Il y a des  attitu- des  obscènes, ces  simulacres de coït qui choquent:le regard. Certes, il  n’est pas question d’approuver, encore que l'insertion de  tels gestes dans des ballets leur confère un  aspect esthétique et  sym- bolique qui  manque aux scènes intimes que le cinéma étale quoti- diennement sous les  yeux des spectateurs (...) Et  le  nu  ? Qu’en dire sinon qu’il est  bref et  de  peu d’effet scénique. Je ne  dirai pas qu'il est  chaste ou  innocent, car  rien de  ce  qui  est  commercial ne peut être ainsi qualifié. Disons qu’il n’est pas choquant et  qu’on peut surtout lui  reprocher d’avoir servi de  miroir aux alouettes pour le  succès de  la  pièce (...). Il  manque à  la  nudité de Hair 5—O. Merlin, «Mlle Thibon dans la vision de Marguerite», Le Monde, 9-2-1 065. 
VIII La distinction d’être symbolique» (6). La  négation de  la jouissance inférieure, grossière, vulgaire, vénale, servile, en  un  mot naturelle, qui cons- titue comme tel  le  sacré culturel, enferme l’affirmation de  la supériorité de  ceux qui savent se  satisfaire des  plaisirs sublimés, raffinés, désintéressés, gratuits, distingués, à jamais interdits aux simples profanes. C’est ce  qui fait que l’art et  la consommation artistique sont prédisposés à  remplir, qu’on le  veuille ou  non, qu’on le  sache ou  non, une fonction sociale de  légitimation des différences sociales. 


% Il y a une économie des biens culturels, mais cette économie a une logique
% spécifique qu’il faut dégager pour échapper à l’économisme. Cela en
% travaillant d’abord à établir les conditions dans lesquelles sont produits
% les consommateurs de biens culturels et leur goût, en même temps qu’à
% décrire les différentes manières de s’approprier ceux d’entre ces biens
% qui sont considérés à un moment donné du temps comme des œuvres d’art et
% les conditions sociales de la constitution du mode d’appropriation qui
% est tenu pour légitime.

% Contre l’idéologie charismatique qui tient les goûts en matière de culture
% légitime pour un don de la nature, l’observation scientifique montre que
% les besoins culturels sont le produit de l’éducation.

% --- continue pasting French text here ---

\end{paracol}


\subsection{Politics of taste}




Consent vs coercion (Gramsci’s key move)

Gramsci distinguishes between:

\begin{itemize}
  \item Domination $\rightarrow$ rule by force (police, law, violence)
  \item Hegemony (\href{https://en.wikipedia.org/wiki/Cultural_hegemony}{cultural hegemony}) $\rightarrow$ rule by consent (culture, norms, meaning)
\end{itemize}

Modern capitalist societies rely more on hegemony than repression.

The most effective power is the one that does not appear as power.

\backtotoc




\subsection{Adorno on culture industry}
philosophy of new music
\subsection{Ethnomusicology}
Chinese opera

\subsubsection{Simhra Arom Pygmees}

\href{https://youtu.be/JrajWGcjaUA}{https://youtu.be/JrajWGcjaUA}

\begin{figure}[htbp]
  \centering
  \includegraphics[width=0.4\linewidth]{images/pygmees.jpg}
  \caption{Pygmees (Simha Arom).}
  \label{fig:pygmees}
\end{figure}

\section{identity - travesti}
'I dress up as taliban, as salsa man, as hip hop, as jazzman, as classical musician.'

'rap Iam '

Like Venables.

\section{medieval mystery play}


\backtotoc



\chapter{Cultural studies}

\href{https://en.wikipedia.org/wiki/Cultural_studies}{cultural studies}


\section{Introduction: Defining the Field}

\textbf{Definition [Cultural Studies]:} \textit{A critical, interdisciplinary academic field that investigates how \textbf{culture}---understood broadly as systems of meaning, representation, and practice---creates, maintains, and transforms individual experiences, social relations, and structures of power.}

Cultural Studies emerged from several intellectual traditions in the mid-20th century, particularly in Britain at the \textsc{Birmingham Centre for Contemporary Cultural Studies} (CCCS). It represents a deliberate departure from:
\begin{itemize}
    \item Traditional literary criticism's focus on \textit{high culture}
    \item Anthropological approaches treating culture as static and holistic
    \item Positivist social science methods claiming objective neutrality
\end{itemize}

\section{Core Characteristics \& Principles}

\subsection{Interdisciplinary Nature}
Cultural Studies deliberately crosses disciplinary boundaries, synthesizing approaches from:

\begin{center}
\begin{tabular}{|l|p{0.7\textwidth}|}
\hline
\textbf{Discipline} & \textbf{Key Contributions} \\
\hline
Sociology & Class analysis, social structures, institutions \\
\hline
Literary Theory & Textual analysis, semiotics, narratology \\
\hline
Marxist Theory & Ideology critique, political economy, hegemony \\
\hline
Feminist Theory & Gender, patriarchy, intersectionality \\
\hline
Postcolonial Theory & Colonialism, race, diaspora, hybridity \\
\hline
Media Studies & Audiences, production, digital cultures \\
\hline
Anthropology & Ethnography, everyday practices, material culture \\
\hline
\end{tabular}
\end{center}
\noindent\textit{Interdisciplinary foundations of Cultural Studies}

\subsection{Political Commitment \& Critical Stance}

\textbf{Principle [The Political is Cultural]:} \textit{All cultural phenomena are understood as inherently political---sites where power is exercised, negotiated, and resisted.}

The field maintains an explicit commitment to:
\begin{itemize}
    \item \textbf{Critique} of dominant power relations (capitalism, patriarchy, racism, colonialism)
    \item \textbf{Intervention} in social and political debates
    \item \textbf{Emancipation} and social justice as normative goals
\end{itemize}

\section{Key Theoretical Concepts}

\subsection{Fundamental Binaries \& Their Deconstruction}

Cultural Studies often works through deconstructing binary oppositions:

\[
\begin{array}{ccc}
\mbox{High Culture} & \longleftrightarrow & \mbox{Popular Culture} \\
\mbox{Producer} & \longleftrightarrow & \mbox{Consumer/Audience} \\
\mbox{Text} & \longleftrightarrow & \mbox{Practice} \\
\mbox{Structure} & \longleftrightarrow & \mbox{Agency} \\
\mbox{Domination} & \longleftrightarrow & \mbox{Resistance} \\
\mbox{Global} & \longleftrightarrow & \mbox{Local} \\
\end{array}
\]

\subsection{Central Theoretical Frameworks}

\subsubsection{Hegemony (Gramsci)}
The process by which dominant groups secure consent through cultural means rather than coercion alone:
\[
\mbox{Hegemony} = \mbox{Domination} + \mbox{Consent} + \mbox{Negotiation}
\]

\subsubsection{Circuit of Culture (du Gay et al.)}
Culture operates through interconnected moments:
\[
\mbox{Representation} \to \mbox{Identity} \to \mbox{Production} \to \mbox{Consumption} \to \mbox{Regulation}
\]

\subsubsection{Articulation (Hall)}
The temporary, non-necessary connection between different elements within cultural formations:
\[
\mbox{Ideology} \bowtie \mbox{Social Forces} \bowtie \mbox{Cultural Forms}
\]

\section{Major Schools \& Thinkers}

\subsection{The Birmingham School (\textsc{CCCS})}
The foundational institution establishing Cultural Studies as a distinct field:

\begin{itemize}
    \item \textbf{Richard Hoggart}: \textit{The Uses of Literacy} (1957)---working-class culture
    \item \textbf{Raymond Williams}: \textit{"Culture is ordinary"}; structures of feeling
    \item \textbf{Stuart Hall}: Encoding/decoding model; articulation theory; race and identity
    \item \textbf{Paul Willis}: Ethnography of working-class youth subcultures
\end{itemize}

\subsection{Key Theoretical Influences}

\begin{center}
\begin{tabular}{|l|l|l|}
\hline
\textbf{Thinker} & \textbf{Key Concept} & \textbf{Application in Cultural Studies} \\
\hline
Antonio Gramsci & Hegemony & Analysis of popular consent to power \\
\hline
Michel Foucault & Power/Knowledge & How discourses produce subjects \\
\hline
Roland Barthes & Myth \& Semiotics & Reading cultural texts as sign systems \\
\hline
bell hooks & Intersectionality & Race, class, gender interconnections \\
\hline
Edward Said & Orientalism & Colonial representations and power \\
\hline
Judith Butler & Performativity & Gender as cultural performance \\
\hline
\end{tabular}
\end{center}
\noindent\textit{Major theoretical influences on Cultural Studies}

\section{Methodological Approaches}

Cultural Studies employs diverse, often combined methodologies:

\subsection{Textual Analysis}
Critical reading of cultural \textit{"texts"} (films, advertisements, fashion, architecture, etc.) as structured systems of signs.

\subsection{Ethnography \& Audience Studies}
Investigating how people actually \textbf{use} and \textbf{makes sense} of cultural products in everyday contexts.

\subsection{Historical Materialism}
Situating cultural forms within their specific historical conditions of production and consumption.

\subsection{Discourse Analysis}
Examining how language and representation construct social reality and subject positions.


\href{https://youtu.be/3RpqgsW6aKQ}{Chttps://youtu.be/3RpqgsW6aKQ}


\href{https://youtu.be/eQGievK8Y5Y}{The Fagotts and their friends}  

\href{https://youtu.be/V07FuMzJptk}{The Fagotts and their friends 3}    

\href{https://youtu.be/Cs9NUY_hUTs}{The Fagotts and their friends 2}   

\href{https://future-lives.com/wp-content/uploads/2014/09/FaggotsAndFriends.pdf}{fagot book}

\href{https://youtu.be/kK0Sh_sNRXs}{les oiseaux}

\href{https://youtu.be/PZn1FpVB8b0}{Brecht 4 sous par Ostermeier}

\href{https://youtu.be/1wxCbQ9-lx4}{picture a day like this}

\href{https://youtu.be/EeY6vD4g_HI}{picture - crimp}

\section{Conclusion: Why Cultural Studies Matters}

Cultural Studies remains vital because it insists on asking the crucial questions that other disciplines often avoid:
\begin{itemize}
    \item Whose interests does this cultural form serve?
    \item How are identities (racial, gender, class, national) constructed and naturalized?
    \item Where are the spaces for resistance, negotiation, and alternative imaginaries?
    \item How does culture both reflect \textit{and} shape material social relations?
\end{itemize}

As Stuart Hall famously argued, culture is not a reflective mirror but a \textit{"constitutive force"} in society---it makes the social world as much as it expresses it.


\backtotoc





\chapter{Ted and Friends}

\section{Ted Huffman}
\href{https://en.wikipedia.org/wiki/Ted_Huffman}{Ted Huffman wiki}

poppea


\section{Baker - Venables}

\href{https://youtu.be/a4s4qpaiGb4}{Below the belt}

\section{Baker/Huffman}
\subsection{Lighthouse}
\href{https://bachtrack.com/review-eto-linbury-maxwell-davies-lighthouse}{lighthouse review - Maxell Davies}

\href{https://youtu.be/yLSDnrsso3M}{lighthouse trailer}
\href{https://youtu.be/OFlF7dVyN_I}{2009}

The opera opens with a prologue in which three officers (tenor, baritone and bass) address a board of inquiry. They relate their voyage to the dark lighthouse and the discovery that the crew was missing, but become increasingly nervous answering the questions put to them by the orchestra's French horn and begin to contradict each other on details. Nevertheless, an open verdict is recorded and the trio sing of the ghost's modern robot replacement.

\subsection{4.48}
\href{https://tedhuffman.com/productions/4.48_psychosis}{4.48 Psychosis}

\section{Venables/Huffman}
\subsection{4.48 Psychosis}

\href{https://youtu.be/_JuRgSbeYpc}{trailer Bayerische Theatrakademie}

\href{https://youtu.be/6m6uR3NRNqU}{Alice}

-> verbatim theatre

\href{https://en.wikipedia.org/wiki/4.48_Psychosis}{4.48 Psychosis - Sarah Kane}    
-> mental illness - \href{https://en.wikipedia.org/wiki/Suicide_(Durkheim_book)}{Durkeim le Suicide}


\subsection{The Fagots and their friends}
\subsection{We are the lucky ones}
\href{https://youtu.be/3bTAA299zn8}{trailer}
\href{https://youtu.be/ESQvqANtcfY}{interviews}

\subsection{Denis and Katia}
\href{https://philipvenables.com/2005/01/24/denis-katya/}{Denis and Katia}



\backtotoc







\chapter{Elements of reading list/bibliography}




\section{General}
\begin{itemize}
  \item \href{https://womeninmusic.voices.wooster.edu/wp-content/uploads/sites/123/2017/12/Koskoff-Gender-Power-and-Music.pdf}{\emph{Gender, Power and Music} -- Judith Koskoff}
  \item \href{https://eng296.digitalwcu.org/wp-content/uploads/2018/09/butler-gender-trouble-chapter-1-w-RC-selections.pdf}{\emph{Gender Trouble: Feminism and the Subversion of Identity} -- Judith Butler}
  \item \href{https://youtu.be/-BnAW4NyOak}{Kimberlé Crenshaw on Intersectionality} - see also \href{https://en.wikipedia.org/wiki/Bell_hooks}{Bell Hooks}
  \item \href{https://hiphopandscreens.wordpress.com/wp-content/uploads/2012/09/rose-black-noise-21-63.pdf}{\emph{Black Noise: Rap Music and Black Culture in Contemporary America} -- Tricia Rose}
\end{itemize}

\section{Syllabus}
\begin{itemize}
  \item \href{https://womeninmusic.voices.wooster.edu/wp-content/uploads/sites/123/2017/12/Koskoff-Gender-Power-and-Music.pdf}{\emph{Gender, Power and Music} -- Judith Koskoff}
  \item \href{https://eng296.digitalwcu.org/wp-content/uploads/2018/09/butler-gender-trouble-chapter-1-w-RC-selections.pdf}{\emph{Gender Trouble: Feminism and the Subversion of Identity} -- Judith Butler}
  \item \href{https://youtu.be/-BnAW4NyOak}{Kimberlé Crenshaw on Intersectionality} - see also \href{https://en.wikipedia.org/wiki/Bell_hooks}{Bell Hooks}
  \item \href{https://hiphopandscreens.wordpress.com/wp-content/uploads/2012/09/rose-black-noise-21-63.pdf}{\emph{Black Noise: Rap Music and Black Culture in Contemporary America} -- Tricia Rose}
\end{itemize}


\backtotoc





\end{document}














% \documentclass[12pt]{book}

% % Packages for clickable TOC
% \usepackage[hidelinks]{hyperref}
% \usepackage{setspace}

% \begin{document}

% \frontmatter
% \tableofcontents

% \mainmatter

% \chapter{Gender, Race, and Power in Music: From Stage to Society}

% This course investigates how operatic works construct, negotiate, and challenge gendered power dynamics across historical periods. Through close analysis of libretti, musical rhetoric, and staging practices, students examine how opera has shaped cultural narratives about femininity, masculinity, and authority. Case studies range from early Baroque heroines to modern reinterpretations of canonical roles.

% Special attention is given to the political implications of vocal types, character archetypes, and performance conventions. The course situates opera within broader social debates on identity, representation, and agency. Students explore intersections of class, race, and sexuality as they manifest in operatic storytelling.

% Contemporary productions are analyzed for their reimagining of inherited power structures. Multimedia resources highlight evolving directorial strategies that foreground gender critique. By the end of the course, students gain tools for interpreting opera as both an artistic form and a sociocultural force. Ultimately, the course reveals how the operatic stage mirrors—and sometimes transforms—the power dynamics of society itself.

% \bigskip
% \emph{Every performance is a political act.}

% \chapter{Gender and Power in Music: From Stage to Society}

% This course investigates how operatic works construct, negotiate, and challenge gendered power dynamics across historical periods. Through close analysis of libretti, musical rhetoric, and staging practices, students examine how opera has shaped cultural narratives about femininity, masculinity, and authority.

% The course situates opera within broader social debates on identity, representation, and agency. Students explore intersections of class, race, and sexuality as they manifest in operatic storytelling. Contemporary productions are analyzed for their reimagining of inherited power structures.

% By the end of the course, students gain tools for interpreting opera as both an artistic form and a sociocultural force. Ultimately, the course reveals how the operatic stage mirrors—and sometimes transforms—the power dynamics of society itself.

% \chapter{Cultural Voices: Gender and Storytelling in Music and Media}

% This course explores how gender shapes narrative strategies across music, opera, and contemporary media. Students investigate how storytellers use sound, voice, and visual framing to construct cultural meanings around identity and power.

% Operatic case studies draw particular attention to the Aix-en-Provence Opera Festival as a site of innovative production. Comparative analysis bridges opera with film, television, and digital media to trace evolving modes of representation. Discussions emphasize how marginalized voices challenge dominant narratives through creative expression.

% Students engage with critiques of authorship, spectatorship, and industry structures. By the end of the course, participants gain a nuanced understanding of how cultural voices shape—and are shaped by—gendered storytelling.

% \chapter{Music, Identity, and the Politics of Performance}

% This course explores how the operatic stage serves as a powerful arena for the construction and contestation of identity. Music, libretto, and staging are examined as forces shaping representations of nationality, race, gender, and class.

% We analyze sonic signifiers, vocal traditions, and the politics of the performing body, including legacies of blackface and yellowface and contemporary movements toward ethical casting. Students investigate how opera companies negotiate tradition, relevance, and social responsibility.

% Ultimately, the course presents opera as a living art form deeply engaged in pressing cultural dialogues, revealing that every performance is a political act.

% \chapter{Staging Power: A History of Political Expression in Music and Theatre}

% How have societies used musical drama to wield power? This course answers that question through a historical survey from ancient Greek civic theatre to contemporary popular performance.

% We examine moments where politics and performance converge, including medieval religious drama, absolutist court spectacle, nationalist opera, and modern protest music. Emphasis is placed on the visual, spatial, and institutional contexts of performance, revealing a shared history of artistic and political expression.

% \chapter{Gender and Power in Music: From Stage to Society}

% This course examines how operatic and musical traditions construct, challenge, and negotiate gendered identities. Students analyze representations of femininity, masculinity, and authority across historical and contemporary works.

% Attention is given to vocal typologies, character archetypes, and staging practices as sites of ideological meaning. Case studies include canonical works and modern adaptations, with particular focus on the Aix-en-Provence Festival and the work of director Ted Huffman.

% Students engage with feminist, queer, and intersectional critiques, developing critical scholarly insight into how music mirrors and shapes the politics of gender in society.

% \chapter{Gender and Power in Music: From Stage to Society II}

% This course explores the dynamic interplay between gender, power, and music with a focus on operatic representation. Through analysis of vocal range, orchestration, and libretto, students examine archetypes such as the tragic heroine, the fallen hero, and the authoritarian figure.

% Moving beyond the score, the course considers power relations behind the scenes among composers, patrons, and performers. From Baroque spectacle to contemporary reinterpretation, students explore how feminist and queer perspectives reshape canonical works.

% \chapter{Opera and Beyond: Gender, Performance, and Power}

% This course explores how opera stages questions of gender, identity, and power from its origins to modern performance. Using Aix-en-Provence and its festival as a living laboratory, students analyze live performance as a primary text.

% The course emphasizes the performer’s body, the director’s vision, and the audience’s gaze. From tragic divas to subversive trouser roles, students examine how gender is performed and contested. The inquiry extends beyond the opera house into public spectacle, revealing how art, politics, and identity converge in shared cultural space.

% \end{document}
