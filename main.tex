\documentclass[12pt]{article}

\usepackage{geometry}
\geometry{
    paperwidth=8.5in,
    paperheight=30in,   % long scrolling page
    margin=1in
}

\usepackage{graphicx}
\usepackage{array}
\usepackage{setspace}
\usepackage{hyperref}
\usepackage{enumitem}
\usepackage{titlesec}
\usepackage{helvet}
\renewcommand{\familydefault}{\sfdefault}

\setstretch{1.2}

\titleformat{\section}
  {\Large\bfseries}
  {}
  {0pt}
  {}
  [\vspace{0.4em}\hrule height 1pt \vspace{1em}]
\begin{document}

%----------------------------------------------------------------------------------
% HEADER WITH LOGO + RIGHT-ALIGNED TITLE BLOCK
%----------------------------------------------------------------------------------

\begin{table}[h!]
\begin{tabular}{ m{3cm} m{14cm} }
    \includegraphics[width=5cm]{images/logo.jpg} &
    \begin{flushright}
        \vspace{-0.5em}
        {\Large \textbf{School of Humanities and Social Sciences}} \\[6pt]
        {\Large \textbf{SOC/HIS 307-AIX 1}} \\[12pt]
        {\huge \textbf{Gender, Race and Power in Music: From Stage to Society}} \\[12pt]
        {\Large \textbf{2026 Summer}}
    \end{flushright}
\end{tabular}
\end{table}

\vspace{1.5em}

%----------------------------------------------------------------------------------
\section*{Course Details}
% Aligned course detail fields (Credit Hours, Days, Time)
\begin{tabular}{@{} l l @{}}
\textbf{Credit Hours:} & 3 \\[0.6em]
\textbf{Days:} & Mondays, Wednesdays \\[0.6em]
\textbf{Time:} & 09:00 -- 10:25 \\
\end{tabular}

\vspace{1em}

\noindent\textbf{Prerequisites:}  
This course is suitable for an undergraduate student who has successfully completed at least 3 semesters of college-level coursework. Successful completion of ENG 101 or equivalent, and 6 credit hours above the 100-level (introductory) in art, art history, literature, music, or the humanities are strongly recommended; or instructor permission.

\vspace{1.5em}

%----------------------------------------------------------------------------------
\section*{Instructor Information}

\textbf{Jonathan Bell} \\
Email: \href{mailto:jonathan.bell@iau.edu}{jonathan.bell@iau.edu}

\vspace{1.5em}

%----------------------------------------------------------------------------------
\section*{Course Description}
This course investigates how operatic works construct, negotiate, and challenge gendered power dynamics across historical periods.
Through close analysis of libretti, musical rhetoric, and staging practices, students examine how opera has shaped cultural narratives about femininity, masculinity, and authority.
Case studies range from early Baroque heroines to modern reinterpretations of canonical roles.
Special attention is given to the political implications of vocal types, character archetypes, and performance conventions.
The course situates opera within broader social debates on identity, representation, and agency.
\textbf{Students explore intersections of class, race, and sexuality as they manifest in operatic storytelling}.
Contemporary productions are analyzed for their reimagining of inherited power structures.
Multimedia resources highlight evolving directorial strategies that foreground gender critique.
By the end of the course, students gain tools for interpreting opera as both an artistic form and a sociocultural force.
Ultimately, the course reveals how the operatic stage mirrors—and sometimes transforms—the power dynamics of society itself.

\textbf{every performance is a political act}

This course investigates how operatic works construct, negotiate, and challenge gendered power dynamics \textbf{across historical periods}.


\vspace{1.5em}

%----------------------------------------------------------------------------------
\section*{Course Objectives}

% \begin{itemize}[leftmargin=1.2cm]
%     \item Develop an appreciation for opera, incorporating elements of aesthetic and historical contextualization.
%     \item Identify the historical period of opera compositions (e.g., Baroque, Classical, Romantic, Post-Romantic) by analyzing their musical characteristics.
%     \item Recognize and describe stylistic cues that are characteristic of Italian, German, French, American, and Chinese opera.
% \end{itemize}

\vspace{1.5em}

%----------------------------------------------------------------------------------
\section*{Learning Outcomes}

% \begin{itemize}[leftmargin=1.2cm]
%     \item Students will be able to analyze and interpret the representation of women and femininity in operatic works from various historical periods and cultures.
%     \item Students will attend an opera and watch recordings of iconic operatic scenes that they will review and discuss through both written and oral assignments.
%     \item Critically evaluate the portrayal of gender roles within opera narratives and performance.
%     \item Examine the socio-economic contexts influencing the development and reception of opera.
% \end{itemize}

\vspace{1.5em}

%----------------------------------------------------------------------------------
\section*{Instructional Methods and Activities}

% Whilst no musical training will be required to follow this course, some knowledge about the main characteristics of musical styles will be provided in class, along with the historical context of the source material and libretto. The course is therefore suitable for students who have varied academic backgrounds. Emphasis will be placed on the student’s personal interpretation of the works, supported by their knowledge of the content and context.

\vspace{1.5em}

%----------------------------------------------------------------------------------
\section*{Primary/Required Textbooks and Materials}

\begin{table}[h!]
\centering
\begin{tabular}{ m{5.5cm} m{11.5cm} }

% --- Row 1 ---
\includegraphics[width=5cm]{images/Gender.jpg} &
Gender theorist and philosopher Judith Butler argues that gender is performative, meaning that it is maintained, created or perpetuated by iterative repetitions when speaking and interacting with each other. Butler draws upon many authors in their[c] work, including Jacques Lacan, Sigmund Freud, Michel Foucault, Julia Kristeva, Jacques Derrida, Simone de Beauvoir, Luce Irigaray, Monique Wittig, among others.
\\[0.5em]

% --- Row 2 ---
\includegraphics[width=5cm]{images/kobbe.jpg} &
The opera-lover's bible from its first appearance, it has now been redesigned and extended, numerous existing entries have been completely rewritten, and the book now incorporates some 200 new operas. The total number of works covered is now nearly 500.
Available \href{https://www.gutenberg.org/files/40540/40540-h/40540-h.htm}{here}.
\\[0.5em]

% --- Row 3 ---
\includegraphics[width=5cm]{images/Faggots.jpg} &
Gender theorist and philosopher Judith Butler argues that gender is performative, meaning that it is maintained, created or perpetuated by iterative repetitions when speaking and interacting with each other. Butler draws upon many authors in their[c] work, including Jacques Lacan, Sigmund Freud, Michel Foucault, Julia Kristeva, Jacques Derrida, Simone de Beauvoir, Luce Irigaray, Monique Wittig, among others.
\\[0.5em]

% --- Row 4 ---
\includegraphics[width=5cm]{images/orientalism.jpg} &
The opera-lover's bible from its first appearance, it has now been redesigned and extended, numerous existing entries have been completely rewritten, and the book now incorporates some 200 new operas. The total number of works covered is now nearly 500.
Available \href{https://www.gutenberg.org/files/40540/40540-h/40540-h.htm}{here}.
\\[0.5em]

% --- Row 5 ---
\includegraphics[width=5cm]{images/levinas.png} &
The opera-lover's bible from its first appearance, it has now been redesigned and extended, numerous existing entries have been completely rewritten, and the book now incorporates some 200 new operas. The total number of works covered is now nearly 500.
Available \href{https://www.gutenberg.org/files/40540/40540-h/40540-h.htm}{here}.
\\


% --- Row 5 ---
\includegraphics[width=5cm]{images/bourdieu2.jpg} &
The opera-lover's bible from its first appearance, it has now been redesigned and extended, numerous existing entries have been completely rewritten, and the book now incorporates some 200 new operas. The total number of works covered is now nearly 500.
Available \href{https://www.gutenberg.org/files/40540/40540-h/40540-h.htm}{here}.
\\

\end{tabular}
\end{table}





\section*{Weekly schedule (12 weeks) - - too historical...}


\begin{itemize}[leftmargin=*]
    \item \textbf{Week 1 — Introduction: Performance, Politics, and Picture-Making} \\
    Topics: definitions, methods; art history as companion discipline (iconography, public art). \\
    Readings: short methodologies on performance studies + visual culture.

    \item \textbf{Week 2 — Greek Tragedy: Civic Ritual and Political Theatre} \\
    Close materials: Aeschylus \textit{Oresteia} (selections), Sophocles \textit{Antigone} (selections). \\
    Focus: chorus vs. protagonist — collective voice as civic conscience; linkage to Classical sculpture and civic monuments. \\
    Key secondary reading on political reception of tragedy.

    \item \textbf{Week 3 — Roman Pantomime, Late Antique Spectacle, and Visual Propaganda} \\
    Topics: spectacle, processional imagery, imperial portraiture; how pageantry projects authority.

    \item \textbf{Week 4 — Liturgical Drama \& Hildegard's \textit{Ordo Virtutum}} \\
    Materials: \textit{Quem quaeritis} trope, \textit{Ordo Virtutum} (listening \& score extracts). \\
    Focus: sacred drama as doctrinal and communal pedagogy; connections to Byzantine icon painting and manuscript illumination.

    \item \textbf{Week 5 — Vernacular Mystery \& Morality Plays (Moyen Âge)} \\
    Case studies: English and French mystery cycles; civic staging as town ritual. \\
    Link to Gothic stained glass and civic façades.

    \item \textbf{Week 6 — Courts, Intermedi, and the Rise of Theatrical Spectacle} \\
    Case: late-15th/16th-century Italian intermedi; court masques. \\
    Focus: court art, allegorical pageantry, perspective stagecraft; how princes used performance to embody rule.

    \item \textbf{Week 7 — Baroque Opera \& Absolutism} \\
    Case studies: Monteverdi, early public opera houses. \\
    Topics: spectacle, set machinery, and absolutist iconography (comparing court portraits and opera scenography).

    \item \textbf{Week 8 — Opera and National Politics (18th–19th centuries)} \\
    Case study: Verdi as national symbol; \textit{Va, pensiero} as political chant; opera houses as national forums. \\
    Discuss opera's visual splendour versus contemporaneous painting (Neoclassicism $\rightarrow$ Romanticism).

    \item \textbf{Week 9 — Musical Theatre Emerges: 20th-century Forms and Politics} \\
    Case studies: Rodgers \& Hammerstein (social themes), \textit{Cabaret}, \textit{Les Misérables}. \\
    Topics: mass culture, propaganda, and theatrical modernism; visual parallels with modern painting and cinema.

    \item \textbf{Week 10 — Contemporary Musical Theatre as Political Forum} \\
    Case: \textit{Hamilton} — casting, musical pastiche, national narrative; discuss staging aesthetics and poster/album art as visual rhetoric. \\
    Readings on musicals \& politics.

    \item \textbf{Week 11 — Pop Stars, Music Videos, and Mass Spectacle (late 20th–21st c.)} \\
    Case studies: Michael Jackson (race, global humanitarianism, protest songs like \textit{They Don't Care About Us}), Rihanna (philanthropy, Caribbean national identity, brand politics), Taylor Swift (voter mobilization, public endorsements, media image). \\
    Discuss how music videos, stadium tours, and celebrity branding use visual art strategies (advertising, fashion photography, monumentality) to carry political messages.

    \item \textbf{Week 12 — Final Symposium: Student Presentations and Coda} \\
    Students present research; collective discussion on future trajectories (AI, streaming, protest music).
\end{itemize}
\vspace{1.5em}


\end{document}
